% Generated by ChatGPT
\documentclass{article}

\usepackage{tikz}
\begin{document}
\pagestyle{empty}

\def\layersep{2.5cm}

\begin{tikzpicture}[shorten >=1pt,->,draw=black!50, node distance=\layersep]
    \tikzstyle{every pin edge}=[<-,shorten <=1pt]
    \tikzstyle{neuron}=[circle,fill=black!25,minimum size=17pt,inner sep=0pt]
    \tikzstyle{input neuron}=[neuron, fill=green!50];
    \tikzstyle{output neuron}=[neuron, fill=red!50];
    \tikzstyle{hidden neuron}=[neuron, fill=blue!50];
    \tikzstyle{annot} = [text width=4em, text centered]

    % Draw the input layer nodes with adjusted y-coordinates
    \foreach \name / \y in {1,2,3,4}
        \node[input neuron, pin=left:Input \#\name] (I-\name) at (0,\y-11) {};

    % Draw the first hidden layer nodes (10 nodes) with adjusted y-coordinates
    \foreach \name / \y in {1,...,10}
        \path[yshift=-3cm]
            node[hidden neuron] (H1-\name) at (\layersep,-\y cm) {};

    % Draw the second hidden layer nodes (20 nodes)
    \foreach \name / \y in {1,...,20}
        \path[yshift=0.5cm]
            node[hidden neuron] (H2-\name) at (2*\layersep,-\y cm) {};

    % Draw the output layer nodes with adjusted y-coordinates
    %\foreach \name / \y in {1,2,3}
    \node[output neuron,pin={[pin edge={->}]right:Iris Setosa}, right of=H2-10] (O-1) at (2*\layersep,1-10.5) {};
    \node[output neuron,pin={[pin edge={->}]right:Iris Versicolour}, right of=H2-10] (O-2) at (2*\layersep,2-10.5) {};
    \node[output neuron,pin={[pin edge={->}]right:Iris Virginica}, right of=H2-10] (O-3) at (2*\layersep,3-10.5) {};

    % Connect input layer to the first hidden layer
    \foreach \source in {1,2,3,4}
        \foreach \dest in {1,...,10}
            \path (I-\source) edge (H1-\dest);

    % Connect the first hidden layer to the second hidden layer
    \foreach \source in {1,...,10}
        \foreach \dest in {1,...,20}
            \path (H1-\source) edge (H2-\dest);

    % Connect the second hidden layer to the output layer
    \foreach \source in {1,...,20}
        \foreach \dest in {1,2,3}
            \path (H2-\source) edge (O-\dest);

    % Annotate the layers
    \node[annot, above of=H1-1, node distance=1cm] {Hidden Layer 1 (10 nodes)};
    \node[annot, above of=H2-1, node distance=1cm] {Hidden Layer 2 (20 nodes)};
\end{tikzpicture}
% End of code
\end{document}

